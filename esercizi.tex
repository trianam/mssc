
%%% Local Variables: 
%%% mode: latex
%%% TeX-master: t
%%% End: 

\documentclass[a4paper,twosides]{report}
\usepackage[utf8]{inputenc}
\usepackage{lmodern}
\usepackage[T1]{fontenc}
\usepackage[italian]{babel}
\usepackage{microtype}
\usepackage{acronym}
\usepackage{mathtools}
\usepackage{amsfonts}
\usepackage{amsthm}
\usepackage[hidelinks,breaklinks=true]{hyperref}
\usepackage{xcolor}
\usepackage{listings}

\newcommand{\betaReduction}{\ensuremath{\longrightarrow_{\beta}}}
\newcommand{\betaReductionS}{\ensuremath{\Longrightarrow_{\beta}}}

\author{
  {\Large Stefano Martina}\\
  {\small stefano.martina@stud.unifi.it}\\
  Universit\`a degli Studi di Firenze\\
  Scuola di Scienze Matematiche, Fisiche e Naturali\\
  Corso magistrale di Informatica
}
\title{{\Huge\bfseries Modelli di Sistemi Sequenziali e
    Concorrenti}\\{\large\bfseries esercizi}}

\begin{document}
\maketitle

\section*{3.15}

\section*{4.11}

\section*{5.5}
Dati:
\begin{eqnarray*}
  \mathbf{S}&\equiv&\lambda xyz.xz(yz)\\
  \mathbf{K}&\equiv&\lambda xy.x\\
  \mathbf{I}&\equiv&\lambda x.x
\end{eqnarray*}
mostrare che $\mathbf{SK}=\mathbf{KI}$.

Valgono i passaggi:
\begin{eqnarray*}
  \mathbf{SK}&=&(\lambda xyz.xz(yz))(\lambda xy.x)\\
  &\betaReduction&\lambda yz.(\lambda xy.x)z(yz)\\
  &\betaReduction&\lambda yz.(\lambda y.z)(yz)\\
  &\betaReduction&\lambda yz.z\\
  &\equiv&\lambda xy.y
\end{eqnarray*}
e anche:
\begin{eqnarray*}
  \mathbf{KI}&=&(\lambda xy.x)(\lambda x.x)\\
  &\betaReduction&\lambda y.(\lambda x.x)\\
  &\equiv&\lambda yx.x\\
  &\equiv&\lambda xy.y\\
\end{eqnarray*}
quindi la tesi \`e dimostrata perch\`e i due termini riducono alla
stessa forma. 
\section*{6.10}

\section*{7.6}

\section*{8.11}

\section*{9.7}

\section*{11.10}

\section*{12.1}

\section*{12.2}
 
\end{document}

